\documentclass[amsmath,amssymb,aps,prd,10pt,twocolumn,showkeys]{revtex4}
\usepackage{graphicx}
\usepackage{mathtools}
\usepackage{verbatim}
\DeclareMathOperator\erfc{erfc}
\DeclareMathOperator\erf{erf}
\DeclareMathOperator{\sgn}{sgn}
\DeclareMathOperator{\snr}{SNR}
\begin{document}

\title{Responses to Reviewer Comments: Complex Analysis of Askaryan Radiation: UHE-$\nu$ Identification and Reconstruction using the Hilbert Envelope of Observed Signals}

\author{Jordan C. Hanson}
\email{jhanson2@whittier.edu}
\affiliation{Department of Physics and Astronomy, Whittier College}
\author{Raymond Hartig}
\affiliation{Department of Physics and Astronomy, Whittier College}
\date{\today}

\begin{abstract}
We would like to thank the reviewer for helpful feedback.  We have incorporated all the feedback into this second draft, and we have itemized the changes below.  We apologize for the delayed nature of our response.  We received the feedback just as the corresponding author was moving homes, grading final projects and exams, and traveling for the holidays.  In addition, we had to submit another paper related to a grant just as the Spring 2026 semester began.  We finally have had time to go over these changes, and we hope they are satisfactory.
\end{abstract}

\maketitle

\begin{itemize}
\item ``On Page 6, please provide reference to `and the dielectric properties of the South Pole', and do these dielectric properties include attenuation and birefringence?  I think the sentence should explicitly say what the dielectric properties describe?'' \textit{By dielectric properties, we meant the attenuation length versus frequency and depth.  This effect is included within NuRadioMC.  We examined the documentation of NuRadioMC, and found the portion responsible for setting the attenuation versus depth and frequency.  We included two references that first provided data on the attenuation versus frequency and depth at the South Pole.  This text has been included in the second paragraph of Section IV.  As for birefringence, this is not included in NuRadioMC by default.  We felt it was outside the scope of the work.  To include it, we would have to assume two copies of our model in the fit to observed waveforms from NuRadioMC (see further replies below).}
\item ``On Page 6, what does (3K) refer to?  Please explain the relevance to noiseless data.''  \textit{The number 3 Kelvin refers to the noise temperature.  We included it in Table II in the original draft, and now we have added some clarifying remarks in the second and third paragraph of Section IV.  We were incorrect to state that a noise temperature of 3 K makes the data noiseless.  Rather, it makes the noise negligible relative to the UHE-$\nu$ signals.}
\item ``Section IV: Applications to string of closely spaced dipoles operating to produce coherent signals.  Can this procedure also apply to individual dipoles located in a volume of ice?  Can it apply to other antenna systems located at the surface such as in RNO-G design?  More generally, can the authors comment on the applicability to other antenna styles and configurations and insert those comments into the text?''  \textit{In short, yes, this technique does apply to other situations besides the dipole phased array.  We have added the final paragraph of Section IV in the second draft to address the details.  Our technique does apply to RNO-G.  Specifically, we write that ``The only requirement for the RF channel is that $r(t)$ from Sec. III represents a good model of the impulse response, and that suitable values for $f_0$ and $\gamma$ can be measured.}
\item ``The authors presented a compelling case that the predicted shape of the antenna waveforms from the analytic model agree with NuRadioMC.  Do the authors believe similar agreement will be produced by the analytical calculation at higher energies?  Please comment.'' \textit{We are delighted that the reviewer found the case compelling.  We studied the correlations between Eqs. 51-53 from CSWs derived from UHE-$\nu$ events with energies between 10 PeV and 100 PeV, and found good agreement.  In our previous publication (HH2022), we showed good agreement at 100 PeV for $\vec{E}$-fields from hadronic cascades, and for $\vec{E}$-fields from electromagnetic cascades.  By focusing on hadronic showers and not electromagnetic showers at 100 PeV, we intended to avoid discussions of the LPM effect.  This was mostly because we were comparing $\vec{E}$-fields, and not CSWs from voltage traces.  However, in this work, we left the LPM effect activated, and found good agreement up to 100 PeV for all cascade types.  We agree that we should study energies above 100 PeV.  However, we are trying hard to focus the scope of the paper, since there are already several new directions to take the work.  These include: accounting for reflected signals, energies above 100 PeV, on-cone events, separately fitting $\Delta\theta$ ... We will add energies above 100 PeV to the list!}
\item ``Section IV: Does the simulated RF trigger correspond to a trigger rate of 1 Hz?  If so, maybe mention that since the rate is assumed in the next paragraph.'' \textit{We are grateful for this comment, since it reminded us that we can, in fact, predict the trigger rate given a trigger scheme.  We have now included the fifth and sixth paragraphs of Section IV of the second draft.  We write ``We created a computational tool to estimate the thermal trigger rate.  With a majority logic of 3 of 8, a dead-time of 10 ms, a gate time of 200 ns (the default setting in NuRadioMC), sampling rate of 1 GHz, and thresholds of $\approx \pm 3.5 v_{\rm rms}$, we found that Gaussian white noise triggered at $\approx 1$ Hz.''  In the sixth paragraph, we write ``Though we found that such thermal trigger rates allow just 0.2 thermal noise events to pass the correlation threshold in 5 years, this result can be scaled to the appropriate thermal rate ... ''}
\item ``Figures 1 and 2: Do the curves correspond to analytical signal amplitude and signal envelope introduced in Section II?  If so, can one or more of these symbols be included on the y-axis so it is clear to the reader, or at least mentioned in the Figure caption (if appropriate)?''  \textit{This has been done for Figures 1 and 2, in the captions.}
\item ``What is $c_{\rm ave}$ in Table III, and why is the unit length?  I thought it might be $x_{\rm ave}$, but 0.85 is not the average of 0.8 and 0.93. \textit{This was an error on our part.  It should have read $x_{\rm ave}$, but we had used $c$ in earlier notation and calculations.  It was meant to be the geometric error of $x_{\rm em}$ and $x_{\rm had}$, rounded to the nearest bin, but honestly the technique did not work well.  We agree with the reviewer's next comment that Section V in the first draft should be removed.}
\item ``Section V: Traditional methods ... recommend deleting it.''  \textit{In short, we agree.  The reviewer comment is convincing.  It was an original goal of the work to use the $a$-value to find $\log_{10} E_{\rm C}$, as we were able to do in HH2022.  We made our best attempt, and found we were unable to overcome the large fractional error in $\log_{10} E_{\rm C}$ with precision in $\sigma_t$ because we do not yet have a separate fit for $\Delta\theta$.  We recommend in Section VI that one future direction of this work is to create a separate fit for $\Delta\theta$.  With measurements of $\Delta\theta$ and $\sigma_t$, we hope to improve precision in the $a$-value.  In Section V of our second draft, we have made the following changes.  We have removed Table III, much of the text, and discussed why the technique needs improvement.  One key culprit is reflected signals.  In the final paragraph of Section V, we write that ``when reflected signals are present in the CSW, the fitting algorithm compensates with unphysical $\sigma_t$ values corresponding to a single pulse ... This compensation leads to an overestimation of $\log_{10} E_{\rm C}$.}
\item ``Finally, I think the authors may have missed an opportunity to help the reader understand the utility and benefits of this approach. ... \textit{We listed three benefits of fully analytic Askaryan models in HH2022, and we chose to simply cite them in the second paragraph of Section II our second draft.  The main benefit described in this work is thermal noise rejection.  The analytical CSW is so different fom thermal noise triggers, it rejects up to 5 years' worth of noise data in favor of 99.99 percent of UHE-$\nu$ signal.  The reviewer raises an interesting point about RF channel optimization.  That is a use case we should consider in the future.  There has always been a tension between dispersive and non-dispersive RF antennas installed in detectors like RNO-G, for example.  To study this, we could widely vary $\gamma$ while holding other parameters constant.}
\item ``Misspellings: vacuum, just below Equation 3.'' \textit{Thanks for pointing it out, we have fixed this.}
\end{itemize}

\end{document}

