\documentclass{beamer}
\usetheme{metropolis}
\usepackage{graphicx}
\usepackage{hyperref}
\title{Complex analysis of Askaryan Radiation: UHECR and UHE-$\nu$ Reconstruction with Analytic Signals}
\author{Jordan Hanson}
\institute{Whittier College Department of Physics and Astronomy}
\bibliographystyle{abbrv}

\begin{document}
\maketitle

\section{Summary}

\begin{frame}{UHECR and UHE-$\nu$ Reconstruction with Analytic Signals}
\footnotesize
\begin{enumerate}
\footnotesize
\item \alert{\textbf{Introduction}}: 
\footnotesize
\begin{itemize}
\footnotesize
\item \textbf{A fully analytic Askaryan E-field} model in the time-domain \cite{hanson2022complex-5f5}
\item Based on work begun with Prof. Amy Connolly \cite{hanson2017complex-037}
\item Based on work begun by J. Ralston and R. Buniy \cite{buniy2001radio-221}
\item Advantages: (i) extract UHE-$\nu$ cascade parameters by fitting model to raw voltage traces, (ii) fast and simple, (iii) analytic equations may be embedded as event filter
\end{itemize}
\item \alert{\textbf{UHECR and UHE-$\nu$ identification}}:
\begin{itemize}
\footnotesize
\item \textbf{A fully analytic Askaryan voltage} model, time-domain [this work]
\item Equations for both voltage trace, and envelope of voltage trace
\item Verification with NuRadioMC: strong thermal background rejection, signal identification, and (rough) $\log_{\rm 10}(E_{\rm \nu})$ estimate
\end{itemize}
\item \alert{\textbf{Correlation with Recent ARA observations of UHECR}}
\end{enumerate}
\end{frame}

\section{Bibliography}

\begin{frame}{Bibliography}
\bibliography{references}
\end{frame}

\end{document}
