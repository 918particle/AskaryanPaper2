\documentclass{beamer}
\usetheme{metropolis}
\usepackage{graphicx}
\usepackage{hyperref}
\usepackage{tcolorbox}
\DeclareMathOperator\erfc{erfc}
\DeclareMathOperator\erf{erf}
\DeclareMathOperator{\sgn}{sgn}
\DeclareMathOperator{\snr}{SNR}
\definecolor{box_background}{RGB}{240,240,240}
\definecolor{box_frame}{RGB}{50,50,50}
\title{Complex analysis of Askaryan Radiation: UHECR and UHE-$\nu$ Reconstruction with Analytic Signals}
\author{Jordan Hanson}
\institute{Whittier College Department of Physics and Astronomy}
\bibliographystyle{abbrv}

\begin{document}
\maketitle

\section{Summary}

\begin{frame}{UHECR and UHE-$\nu$ Reconstruction with Analytic Signals}
\footnotesize
\begin{enumerate}
\footnotesize
\item \alert{\textbf{Introduction}}: 
\footnotesize
\begin{itemize}
\footnotesize
\item \textbf{A fully analytic Askaryan E-field} model in the time-domain \cite{hh2022}
\item Based on work begun with Prof. Amy Connolly \cite{jchac2017}
\item Based on work begun by J. Ralston and R. Buniy \cite{rb}
\item Advantages: (i) extract UHE-$\nu$ cascade parameters by fitting model to raw voltage traces, (ii) fast and simple, (iii) analytic equations may be embedded as event filter
\end{itemize}
\item \alert{\textbf{UHECR and UHE-$\nu$ identification}}:
\begin{itemize}
\footnotesize
\item \textbf{A fully analytic Askaryan voltage} model, time-domain [this work]
\item Equations for both voltage trace, and envelope of voltage trace
\item Verification with NuRadioMC: strong thermal background rejection, signal identification, and (rough) $\log_{\rm 10}(E_{\rm \nu})$ estimate
\end{itemize}
\item \alert{\textbf{Correlation with Recent ARA observations of UHECR}}
\end{enumerate}
\end{frame}

\section{UHECR Event Geometry for in the ARA Detector}

\begin{frame}{UHECR Event Geometry for in the ARA Detector}
\begin{columns}
\begin{column}{0.6\textwidth}
\begin{figure}
\centering
\includegraphics[width=\textwidth]{figures/ara.png}
\caption{\label{fig:1} (Top right) UHECR cascade core interacting in ice. (Bottom left) ARA RF detection channels (Vpol and Hpol).}
\end{figure}
\end{column}
\begin{column}{0.5\textwidth}
\alert{\textbf{ARA detector schematic}}
\begin{itemize}
\item Askaryan component of the radiation
\item Vpol and Hpol channels are dipole antennas
\item E-field components are radial to the cascade axis
\item Reference: \url{https://arxiv.org/pdf/2510.21104}
\end{itemize}
\end{column}
\end{columns}
\end{frame}

\section{Notations, Definitions, and Analytic Signals}

\begin{frame}{Notations, Definitions, and Analytic Signals}
\footnotesize
\begin{tcolorbox}[colback=box_background,colframe=box_frame,title={Askaryan electric field, $\vec{E}(r,t)$, [V m$^{-1}$]}]
\begin{equation}
r \vec{E}(t,\theta) = -\frac{E_0 \omega_0 \sin(\theta)}{8 \pi p} t_r e^{-\frac{t_r^2}{4p} + p \omega_0^2}\erfc(\sqrt{p}\omega_0)
\end{equation}
\end{tcolorbox}
\begin{table}
\scriptsize
\begin{tabular}{| c | c | c |} \hline
\textbf{Variable} & \textbf{Definition} & \textbf{Units}\\ \hline
$c$ & speed of light in medium & m ns$^{-1}$ \\ 
$r$ & distance to cascade peak & m \\
$t_{\rm r}$ & $t-r/c$ & ns \\
$\theta_{\rm C}$ & Cherenkov angle & radians \\ 
$\theta$ & viewing angle from cascade axis & radians \\ 
$a$ & longitudinal cascade length (see \cite{rb,jchac2017,hh2022}) & m \\ 
$n_{max}$ & max excess cascade particles (see \cite{rb,jchac2017,hh2022})  & none \\
$E_{\rm 0}$ & $\propto n_{\rm max}a$ (see \cite{rb,jchac2017,hh2022}) & V GHz$^{-2}$ \\
$p$ & $\frac{1}{2}(a/c)^2 \left(\cos\theta - \cos\theta_C\right)^2$ (see \cite{hh2022}) & ns$^2$ \\ 
$\omega_{\rm 0}$ & $\sqrt{\frac{2}{3}} (c\sqrt{2\pi}\rho_{\rm 0})/(\sin\theta)$ (see \cite{jchac2017,hh2022}) & GHz \\
$\sqrt{2\pi}\rho_{\rm 0}$ & lateral ICD width (see \cite{jchac2017,hh2022}) & m$^{-1}$ \\ \hline
\end{tabular}
\caption{\label{tab:param} Parameters relevant for $E(t)$.}
\end{table}
\end{frame}

\begin{frame}{Notations, Definitions, and Analytic Signals}
\footnotesize
\begin{tcolorbox}[colback=box_background,colframe=box_frame,title={Askaryan \textit{signal}, $\vec{s}(t)$, [V m$^{-1}$]}]
Let $E_0$ represent all amplitude factors not dependent on time.  Further, let $\sigma_{\rm t} = \sqrt{2p}$.  The Askaryan E-field is
\begin{equation}
s(t) = -E_0 t e^{-\frac{1}{2}\left(t/\sigma_t\right)^2}
\end{equation}
\end{tcolorbox}
\begin{tcolorbox}[colback=box_background,colframe=box_frame,title={Detector \textit{response}, $\vec{r}(t)$, [m ns$^{-1}$]}]
\begin{equation}
r(t) = R_0 e^{-2 \pi \gamma t} \cos(2\pi f_0 t)
\end{equation}
\end{tcolorbox}
\begin{itemize}
\item Impulse response of a causal ($t\geq 0$) damped, driven RLC circuit
\item First used by the RICE collaboration \cite{10.1103/PhysRevD.85.062004}
\item RF dipole channels are used in RICE, ARA, RNO-G, and the proposed IceCube Gen2 (radio) \cite{10.1103/PhysRevD.85.062004,10.1103/physrevd.102.043021,10.1016/j.astropartphys.2011.11.010,10.1088/1748-0221/16/03/p03025,10.48550/arxiv.2008.04323}
\item RF channels must fit inside cylindrical ice boreholes
\end{itemize}
\end{frame}

\begin{frame}{Notations, Definitions, and Analytic Signals}
\footnotesize
\begin{tcolorbox}[colback=box_background,colframe=box_frame,title={Analytic signal, $s_a(t)$, of $s(t)$.}]
Let $s_a(t)$ be the \textit{analytic signal} of the signal $s(t)$.  Further, let $\hat{s}(t)$ be the Hilbert transform of $s(t)$.  Finally, let $\mathcal{E}_s(t)$ be the \textit{envelope} of $s(t)$.
\begin{align}
s_a(t) &= s(t) + j\hat{s}(t) \\
\mathcal{E}_s(t) &= |s_a(t)|
\end{align}
\end{tcolorbox}
\begin{tcolorbox}[colback=box_background,colframe=box_frame,title={Analytic signals, $s_a(t)$, and $r_a(t)$.}]
Let $x = t/(\sqrt{2}\sigma_{\rm t})$, and let $D(x)$ be the \textit{Dawson function.}  The analytic signals for $s(t)$ (V m$^{-1}$) and $r(t)$ (m ns$^{-1}$) are
\begin{align}
s_a(t) &= -E_0 \left(t e^{-\frac{1}{2}\left(t/\sigma_t\right)^2} - \frac{2 j\sigma_t}{\sqrt{2\pi}} \frac{dD(x)}{dx}\right)  \\
r_a(t) &= R_0 e^{-2 \pi \gamma t} e^{2\pi j f_0 t}
\end{align}
\end{tcolorbox}
\end{frame}

\begin{frame}{Notations, Definitions, and Analytic Signals}
\begin{tcolorbox}[colback=box_background,colframe=box_frame,title={Detected signals, $r(t) * s(t)$, [V]}]
\begin{equation}
r(t) * s(t) = \int_{-\infty}^{\infty} r(\tau) s(t-\tau) d\tau
\end{equation}
\end{tcolorbox}
\begin{tcolorbox}[colback=box_background,colframe=box_frame,title=Theorem: the envelope of detected signal]
Let $\mathcal{E}_{r * s}(t)$ represent the \textit{envelope} of the convolution of $r(t)$ and $s(t)$.  If $s_a(t)$ and $r_a(t)$ are the analytic signals of $s(t)$ and $r(t)$, respectively, then
\begin{equation}
\mathcal{E}_{r*s}(t) = \frac{1}{2}|r_a(t) * s_a(t)|
\end{equation}
\end{tcolorbox}
\end{frame}

\begin{frame}{Notations, Definitions, and Analytic Signals}
\begin{tcolorbox}[colback=box_background,colframe=box_frame,title=$\mathcal{E}_{r*s}(t)$: part I]
Let $x=t/(\sqrt{2}\sigma_t)$, $y=\tau/(\sqrt{2}\sigma_t)$, and $z = (2\pi j f_0 - 2\pi\gamma)\sqrt{2}\sigma_t$. Let $w(q)$ be the \textit{Faddeeva function,} with $b = jq$, $b = x+\frac{1}{2} z$.  The convolution of $r_a(t)$ with $\Re\lbrace s_a(t)\rbrace$ is
\begin{multline}
r_a(t) * \Re\lbrace s_a(t) \rbrace = \\ -\sqrt{\pi} R_0 E_0 \sigma_t^2 \left(x e^{-x^2} w(q) + \left(\frac{j}{2}\right) e^{-x^2} \frac{dw(q)}{dq} \right)
\end{multline}
\end{tcolorbox}
\end{frame}

\begin{frame}{Notations, Definitions, and Analytic Signals}
\begin{tcolorbox}[colback=box_background,colframe=box_frame,title=$\mathcal{E}_{r*s}(t)$: part II]
Let $x=t/(\sqrt{2}\sigma_t)$, $y=\tau/(\sqrt{2}\sigma_t)$, and $z = (2\pi j f_0 - 2\pi\gamma)\sqrt{2}\sigma_t$.  Let $u = x-y$, $z=-k$, and let $\mathcal{L}\lbrace D(u-x)\rbrace_k$ be the Laplace transform of the shifted Dawson function.  The convolution of $r_a(t)$ with $\Im\lbrace s_a(t)\rbrace$ is
\begin{multline}
r_a(t) * \Im\lbrace s_a(t) \rbrace = \\ \frac{2}{\sqrt{\pi}} R_0 E_0 \sigma_t^2 \left(D(x) + k\mathcal{L}\lbrace D(u-x)\rbrace_k\right) \label{eq:Im_result}
\end{multline}
\end{tcolorbox}
\end{frame}

\begin{frame}{Notations, Definitions, and Analytic Signals}
\begin{tcolorbox}[colback=box_background,colframe=box_frame,title=$\mathcal{E}_{r*s}(t)$: part III]
\begin{align}
\mathcal{E}_{r*s}(t) &= \frac{1}{2}| r_a * s_a| = \frac{1}{2}\left| r_a * \left( \Re\lbrace s_a\rbrace + j\Im\lbrace r_a\rbrace\right)\right| \\
\mathcal{E}_{r*s}(t) &= \frac{1}{2}\left| r_a * \Re\lbrace s_a\rbrace + j r_a * \Im\lbrace r_a\rbrace \right|
\end{align}
\end{tcolorbox}
\begin{itemize}
\item We have calculated $r_a(t) * \Re\lbrace s_a(t)\rbrace$ and $r_a(t) * \Im\lbrace s_a(t)\rbrace$
\item Combine to form the final result
\item This result is used to predict the \textit{envelope} of the voltage traces and CSWs
\item Python3 code provided in the paper
\end{itemize}
\end{frame}

\begin{frame}{Notations, Definitions, and Analytic Signals}
\begin{tcolorbox}[colback=box_background,colframe=box_frame,title=$\mathcal{E}_{r*s}(t)$: part IV]
Using prior definitions, it turns out that $s * r$ is
\begin{multline}
s * r = -\sqrt{\pi}R_0 E_0 \sigma_t^2 \\ \Re\left\lbrace xe^{-x^2} w(q) - \frac{1}{2}e^{-x^2} \frac{d w(q)}{dx} \right\rbrace
\end{multline}
\end{tcolorbox}
\begin{itemize}
\item This result is used to predict the voltage traces and CSWs
\item Python3 code provided in the paper
\end{itemize}
\end{frame}

\section{Results: Voltage Traces and Envelopes}

\begin{frame}{Results: Voltage Traces and Envelopes}
\begin{figure}[ht]
\centering
\includegraphics[width=0.49\textwidth]{figures/July3rd_plot1.pdf}
\includegraphics[width=0.49\textwidth]{figures/July3rd_plot2.pdf}
\caption{\label{fig:fig1} (Left) The thin black line represents $s*r$.  The light gray envelope represents the envelope of $s*r$ computed with the Python3 SciPy function scipy.special.hilbert. The black envelope represents $\mathcal{E}_{r*s}(t)$. (Right) Same as left, for different parameter values.}
\end{figure}
\end{frame}

\begin{frame}{Results: Voltage Traces and Envelopes}
\begin{figure}[ht]
\centering
\includegraphics[width=0.49\textwidth]{figures/July7th_plot1.pdf}
\includegraphics[width=0.49\textwidth]{figures/July7th_plot2.pdf}
\caption{\label{fig:fig2} (Left) The thin black line represents $s$ convolved with $r$ using the Python3 SciPy function scipy.signal.convolve. The gray line represents $s*r$. (Right) Same as left, for different parameter values.}
\end{figure}
\end{frame}

\section{Results: Comparisons to NuRadioMC}

\begin{frame}{Results: Comparisons to NuRadioMC}
\begin{table}
\small
\centering
\begin{tabular}{| c | c | c |}
\hline
\textbf{Parameter} & \textbf{Value} & \textbf{Note} \\
Ice Model & South Pole & 2015 measurements \\
Signal Model & AHRZ2020 & (see \cite{PhysRevD.101.083005}) \\
Trigger & 3 of 8 channels & $\pm 3v_{\rm rms}$ \\
RF channels & 8 & RF bicone (in firn) \\
Channel filters & [80-1000] MHz & Passband \\
Noise Temperature & 233K & Sets $v_{\rm rms}$ \\
Sampling Rate & 1 GHz & $f_{\rm c} = 500$ MHz \\
Samples per channel & 256 & total time, $256$ ns \\
Channel depths & [-4,-6,-8,...-18] m & cable delays included \\
RF cable type & LMR-400 & $\approx -1$ dB at 20 m  \\
\hline
\end{tabular}
\caption{\label{tab:1} Important NuRadioMC parameters.}
\end{table}
NuRadioMC was used to generate 15133 UHE-$\nu$ events that triggered the detector with $E_{\rm C} = 100$ PeV.
\end{frame}

\begin{frame}{Results: Comparisons to NuRadioMC}
\small
\alert{\textbf{Bottom line:}} 0.2 noise events pass correlation threshold in 5 years.
\begin{figure}
\centering
\includegraphics[width=0.49\textwidth,trim=3.25cm 8.25cm 4.5cm 9.0cm,clip=true]{figures/Aug15_plot1.pdf}
\includegraphics[width=0.49\textwidth,trim=3.25cm 8.25cm 4.5cm 9.0cm,clip=true]{figures/Aug15_plot2.pdf}
\caption{\small \label{fig:fig3} (Left) The black circles represent the noise distribution.  The gray dashed line is a fitting function to noise distribution.  The solid gray line represents the correlation distribution for mathematical envelope to envelope of UHE-$\nu$ signals.  The dashed black line is the correlation threshold. (Right) The correlation coefficient versus SNR.}
\end{figure}
\end{frame}

\section{Results: ARA5 UHECR Results}

\begin{frame}{Results: ARA5 UHECR Results}
\small
\alert{\textbf{Correlation coefficient}} for ARA5 UHECR candidate and model: \textbf{0.853}.
\begin{figure}
\centering
\includegraphics[width=0.75\textwidth]{figures/November2nd_plot1.pdf}
\caption{\small \label{fig:fig4} Fit for analytic $s*r$ to an ARA5 UHECR candidate 1915-26288.}
\end{figure}
\end{frame}

\begin{frame}{Results: ARA5 UHECR Results}
\small
\begin{figure}
\centering
\includegraphics[width=0.75\textwidth]{figures/November10th_plot1.pdf}
\caption{\small \label{fig:fig5} Fit for analytic $s*r$ to an ARA5 UHECR candidate 1957-13330.}
\end{figure}
\end{frame}

\begin{frame}{Results: ARA5 UHECR Results}
\small
\begin{figure}
\centering
\includegraphics[width=0.75\textwidth]{figures/November10th_plot2.pdf}
\caption{\small \label{fig:fig6} Fit for analytic $s*r$ to an ARA5 UHECR candidate 2171-31805.}
\end{figure}
\end{frame}

\section{Conclusion}

\begin{frame}{UHECR and UHE-$\nu$ Reconstruction with Analytic Signals}
\footnotesize
\begin{enumerate}
\footnotesize
\item \alert{\textbf{Introduction}}: 
\footnotesize
\begin{itemize}
\footnotesize
\item \textbf{A fully analytic Askaryan E-field} model in the time-domain \cite{hh2022}
\item Based on work begun with Prof. Amy Connolly \cite{jchac2017}
\item Based on work begun by J. Ralston and R. Buniy \cite{rb}
\end{itemize}
\item \alert{\textbf{UHECR and UHE-$\nu$ identification}}:
\begin{itemize}
\footnotesize
\item \textbf{A fully analytic Askaryan voltage} model, time-domain [this work]
\item Equations for both voltage trace, and envelope of voltage trace
\item Verification with NuRadioMC: strong thermal background rejection, signal identification, and (rough) $\log_{\rm 10}(E_{\rm \nu})$ estimate
\end{itemize}
\item \alert{\textbf{Correlation with Recent ARA observations of UHECR}}
\begin{itemize}
\item Three events reconstructed, with ten more to come
\item Improvements: noise filter, channel selection
\item \textbf{Proposal:} short manuscript with equations and fits
\item Mathematical physics and NuRadioMC comparison has been submitted to \textbf{PRD}
\end{itemize}
\end{enumerate}
\end{frame}

\section{Bibliography}

\begin{frame}[allowframebreaks]{Bibliography}
\bibliography{references}
\end{frame}

\end{document}
