\documentclass{beamer}
\usetheme{metropolis}
\usepackage{graphicx}
\usepackage{hyperref}
\usepackage{tcolorbox}
\DeclareMathOperator\erfc{erfc}
\DeclareMathOperator\erf{erf}
\DeclareMathOperator{\sgn}{sgn}
\DeclareMathOperator{\snr}{SNR}
\definecolor{box_background}{RGB}{240,240,240}
\definecolor{box_frame}{RGB}{40,40,40}
\title{Complex analysis of Askaryan Radiation: UHECR and UHE-$\nu$ Reconstruction with Analytic Signals}
\author{Jordan Hanson}
\institute{Whittier College Department of Physics and Astronomy}
\bibliographystyle{abbrv}

\begin{document}
\maketitle

\section{Summary}

\begin{frame}{UHECR and UHE-$\nu$ Reconstruction with Analytic Signals}
\footnotesize
\begin{enumerate}
\footnotesize
\item \alert{\textbf{Introduction}}: 
\footnotesize
\begin{itemize}
\footnotesize
\item \textbf{A fully analytic Askaryan E-field} model in the time-domain \cite{hh2022}
\item Based on work begun with Prof. Amy Connolly \cite{jchac2017}
\item Based on work begun by J. Ralston and R. Buniy \cite{rb}
\item Advantages: (i) extract UHE-$\nu$ cascade parameters by fitting model to raw voltage traces, (ii) fast and simple, (iii) analytic equations may be embedded as event filter
\end{itemize}
\item \alert{\textbf{UHECR and UHE-$\nu$ identification}}:
\begin{itemize}
\footnotesize
\item \textbf{A fully analytic Askaryan voltage} model, time-domain [this work]
\item Equations for both voltage trace, and envelope of voltage trace
\item Verification with NuRadioMC: strong thermal background rejection, signal identification, and (rough) $\log_{\rm 10}(E_{\rm \nu})$ estimate
\end{itemize}
\item \alert{\textbf{Correlation with Recent ARA observations of UHECR}}
\end{enumerate}
\end{frame}

\section{UHECR Event Geometry for in the ARA Detector}

\begin{frame}{UHECR Event Geometry for in the ARA Detector}
\begin{columns}
\begin{column}{0.6\textwidth}
\begin{figure}
\centering
\includegraphics[width=\textwidth]{figures/ara.png}
\caption{\label{fig:1} (Top right) UHECR cascade core interacting in ice. (Bottom left) ARA RF detection channels (Vpol and Hpol).}
\end{figure}
\end{column}
\begin{column}{0.5\textwidth}
\alert{\textbf{ARA detector schematic}}
\begin{itemize}
\item Askaryan component of the radiation
\item Vpol and Hpol channels are dipole antennas
\item E-field components are radial to the cascade axis
\item Reference: \url{https://arxiv.org/pdf/2510.21104}
\end{itemize}
\end{column}
\end{columns}
\end{frame}

\section{Notations and Definitions}

\begin{frame}{Notations and Definitions}
\footnotesize
\begin{tcolorbox}[colback=box_background,colframe=box_frame,title={Askaryan electric field, $\vec{E}(r,t)$, [V m$^{-1}$]}]
\begin{equation}
r \vec{E}(t,\theta) = -\frac{E_0 \omega_0 \sin(\theta)}{8 \pi p} t_r e^{-\frac{t_r^2}{4p} + p \omega_0^2}\erfc(\sqrt{p}\omega_0)
\end{equation}
\end{tcolorbox}
\begin{table}
\scriptsize
\begin{tabular}{| c | c | c |} \hline
\textbf{Variable} & \textbf{Definition} & \textbf{Units}\\ \hline
$c$ & speed of light in medium & m ns$^{-1}$ \\ 
$r$ & distance to cascade peak & m \\
$t_{\rm r}$ & $t-r/c$ & ns \\
$\theta_{\rm C}$ & Cherenkov angle & radians \\ 
$\theta$ & viewing angle from cascade axis & radians \\ 
$a$ & longitudinal cascade length (see \cite{rb,jchac2017,hh2022}) & m \\ 
$n_{max}$ & max excess cascade particles (see \cite{rb,jchac2017,hh2022})  & none \\
$E_{\rm 0}$ & $\propto n_{\rm max}a$ (see \cite{rb,jchac2017,hh2022}) & V GHz$^{-2}$ \\
$p$ & $\frac{1}{2}(a/c)^2 \left(\cos\theta - \cos\theta_C\right)^2$ (see \cite{hh2022}) & ns$^2$ \\ 
$\omega_{\rm 0}$ & $\sqrt{\frac{2}{3}} (c\sqrt{2\pi}\rho_{\rm 0})/(\sin\theta)$ (see \cite{jchac2017,hh2022}) & GHz \\
$\sqrt{2\pi}\rho_{\rm 0}$ & lateral ICD width (see \cite{jchac2017,hh2022}) & m$^{-1}$ \\ \hline
\end{tabular}
\caption{\label{tab:param} Parameters relevant for $E(t)$.}
\end{table}
\end{frame}

\begin{frame}{Notations and Definitions}
\footnotesize
\begin{tcolorbox}[colback=box_background,colframe=box_frame,title={Askaryan \textit{signal}, $\vec{s}(t)$, [V m$^{-1}$]}]
\begin{equation}
s(t) = -E_0 t e^{-\frac{1}{2}\left(t/\sigma_t\right)^2}
\end{equation}
\end{tcolorbox}
\begin{tcolorbox}[colback=box_background,colframe=box_frame,title={Detector \textit{response}, $\vec{r}(t)$, [m ns$^{-1}$]}]
\begin{equation}
r(t) = R_0 e^{-2 \pi \gamma t} \cos(2\pi f_0 t)
\end{equation}
\end{tcolorbox}
\end{frame}

\section{Bibliography}

\begin{frame}[allowframebreaks]{Bibliography}
\bibliography{references}
\end{frame}

\end{document}
